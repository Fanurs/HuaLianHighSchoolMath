\documentclass[a4,12pt]{article}
	\usepackage[UTF8]{ctex}
	\usepackage{datetime2}% to use \DTMnow
	\usepackage{indentfirst}
		\setlength{\parskip}{5pt}% spacing between paragraphs
		\setlength{\parindent}{20pt}% indentation
	\usepackage{geometry}
		\geometry{tmargin=8mm,bmargin=15mm,lmargin=8mm,rmargin=8mm}
	\usepackage{enumitem}
	\usepackage{amsmath}
	\usepackage{amssymb}% use mathcal, etc.
	\usepackage{mathtools}
	\usepackage{physics}
	\usepackage{tabu}
	\usepackage{tikz}
		\usetikzlibrary{arrows.meta,calc,decorations.markings,math,arrows.meta,patterns,angles,quotes,decorations.pathreplacing}

% self-defined commands always begin with UPPERCASE LETTER
	\newenvironment{LatinModern}{\fontfamily{lmr}\selectfont}{\par}% Latin Modern
	\DeclareTextFontCommand{\textmyfont}{\LatinModern}
	\newcommand{\Curs}[1]{\emph{\LatinModern{#1}}}
	\newcounter{Problem}
	\newcommand{\Problem}[1]{
		\vspace*{10pt}
		\stepcounter{Problem}
		\label{Problem \arabic{Problem}}
		\noindent\arabic{Problem}.\emph{~#1}
	}
	\newcommand{\Qed}{\hfill\ensuremath{\square}}

% useful symbols
	\def\Tick{\ding{51}}
	\newcommand{\rectangle}{{
		\ooalign{$\sqsubset\mkern3mu$\cr$\mkern3mu\sqsupset$\cr}
	}}

\begin{document}

% Title
	\title{
		\vspace*{-50pt}
		\Large{\textbf{太平华联中学数学竞赛题解}\vspace*{-10pt}}\\
	}
	\author{作者:郑其恩 Fanurs\vspace*{-30pt}}
	\date{最后编译时间:\DTMnow (美东)}
	\maketitle
	\vspace*{-25pt}

\vspace*{10pt}\hrule\vspace*{10pt}
\begin{center}\Large{\emph{今日内容:二元一次不定方程与辗转相除法}}\end{center}

昨日的题解中,其中一题其实可以用“不定方程”的知识更轻松地解答,而不定方程也是竞赛中最热门的题型之一,所以今天作者将系统性地介绍不定方程的基本解题技巧。

不定方程,又称为丢番图方程(Diophantine equation),是未知数$x$要求为整数(可正可负可零)的整数系数多项式等式。其一般表达式为
\[ a_1x_1^{n_1} + a_2x_2^{n_2} + \cdots + a_kx_k^{n_k} = c \ \mbox{,} \]
其中所有系数$a_1, a_2, \ldots, a_k$,所有幂$n_1, n_2, \ldots, n_k$以及$c$均为整数。如果能找到至少一组整数解,即存在着至少一组$r_1, r_2, \ldots, r_k\in\mathbb{Z}$,使得
\[ a_1r_1^{n_1} + a_2r_2^{n_2} + \cdots + a_kr_k^{n_k} = c \ \mbox{,} \]
我们便说此不定方程有解;反之,我们便说此不定方程无解。

不定方程,时至今日,都是非常困难的问题。著名的“费马大定理”便是一种不定方程,该猜想(被证明之前被称为“费马猜想”)最先由法国数学皮埃尔·德·费马于1637年提出,却一直等到了358年后的1995年,才被英国数学家安德鲁·约翰·怀尔斯爵士成功证明。竞赛中出现的不定方程虽然比这简单得多,但也足以给很多参赛同学带来不小的挑战。

\paragraph{二元一次不定方程}
	中学竞赛中出现的不定方程,很多时候都只是一次不定方程,即所有幂为一的不定方程,其一般表达式为
	\[ a_1x_1 + a_2x_2 + \cdots + a_kx_k = c \ \mbox{。} \]
	而多元一次不定方程又是建立在二元一次不定方程,
	\[ a_1x_1 + a_2x_2 = c \ \mbox{。} \]
	当只有两个未知数,我们一般更倾向把它们写作$x$和$y$,对应的系数也习惯改成$a$和$b$。本文接下来所有的讨论都将依据以下二元一次不定方程的一般式:
	\[ ax + by = c \ \mbox{。} \]

\paragraph{一个可解的例子}
	首先我们先来看一个具体的例子,
	\[ 2x+3y=5 \ \mbox{。} \]
	显然,大家都应能直接看出$(x, y) = (1, 1)$是此方程的一个解。但这解会是唯一的吗?如果不把这看作不定方程,即允许非整数解,显然我们知道存在着无限多个解。这是因为我们能任意选个$x$值,比如$x:=t\in\mathbb{R}$,那么$y$便直接等于
	\[ y = \frac{5-2t}{3} \ \mbox{。} \]
	比如,我们可以选$t:=2$,这样就有$(x, y) = (2, 1/3)$。然而,如果仅考虑整数解,$(2, 1/3)$便不能被接受了。倘若继续试错,数感强的读者或许能很快得到其它的整数解,比如$(x, y) = (-2, 3), (4, -1)$等等。但是这样猜出来的答案,当遇到数字大点的题目,会显得很无力。因此接下来,我们将学习不依赖数感和大量试错的方法。

	首先,我们先假设$(x, y) = (1, 1)$是我们唯一猜得出来的解。接下来为了得到其它满足此方程的解,我们利用如下的技巧:
	\[ \begin{aligned}
			2x + 3y &= 5 \\
			2x  +3y + (2\times3) - (2\times3) &= 5 \\
			(2x + 2\times3) + (3y - 2\times3) &= 5 \\
			2(x+3) + 3(y-2) &= 5 \\
			2x_1 + 3y_1 &= 5 \ \mbox{。}
		\end{aligned}
	\]
	由此可见,如果$(x, y)$是方程的解,那么$(x_1, y_1) = (x+3, y-2)$也会是方程的解,比如我们已经知道$(x,y)=(1,1)$是一个解,而这“新鲜出炉的发现”便告诉我们另一组解为
	\[(x_1, y_1) = (1+3, 1-2) = (4, -1) \ \mbox{。} \]
	我们可以很快地把$(x_1, y_1) = (4, -1)$代入回原方程,发现的确是个解。这个方法巧妙之处在于还能递归式使用,也就是如果把$(4, -1)$重新以$(x, y)$代入,那又能生成一组新的解,$(4+3, -1-2) = (7,-3)$。以此类推,我们可以很快得到
	\[ (x, y) = (1,1), (4,-1), (7,-3), (10,-5), (13,-7), \ldots \ \mbox{。} \]
	但这些就是方程的所有解了吗?还不是的。

	上述所用到的“$+ (2\times3) - (2\times3)$”技巧,若添加个可调参数$t\in\mathbb{Z}$,并设$(x_0, y_0)$为不定方程的其中一组已知的解,则
	\[ \begin{aligned}
			2x_0 + 3y_0 &= 5 \\
			2x_0 + 3y_0 + (2\times3)t - (2\times3)t &= 5 \\
			(2x_0 + 2\times3t) + (3y_0 - 2\times3t) &= 5 \\
			2(x_0+3t) + 3(y_0-2t) &= 5 \\
			2x + 3y &= 5 \\
		\end{aligned}
	\]
	因此$2x + 3y = 5$的所有解都可以写成
	\[ \begin{dcases}
			x = x_0 + 3t \\
			y = y_0 - 2t
		\end{dcases}
	\]
	的形式,其中$(x_0, y_0)$是方程已知的一个解,在这里我们取之为$(1, 1)$。作者也选了几个$t$值,把其对应的解整理成下表:
	\begin{center}
	\begin{tabular}{c|ccccccccc}
		$t$ & $\cdots$ & $-3$ & $-2$ & $-1$ & $0$ & $1$ & $2$ & $3$ & $\cdots$ \\
		\hline
		$x$ & $\cdots$ & $-8$ & $-5$ & $-2$ & $1$ & $4$ & $7$ & $10$ & $\cdots$ \\
		$y$ & $\cdots$ & $7$ & $5$ & $3$ & $1$ & $-2$ & $-4$ & $-6$ & $\cdots$ \\
	\end{tabular}
	\end{center}
	这一次,我们把$2x+3y=5$所有可能的整数解都按规律列出来了——总共有无限多组整数解。

\paragraph{一个不可解的例子}
	不是每个二元一次不定方程都有解,比如
	\[ 2x + 4y = 5 \]
	就无解。不相信的话,可以多尝试几组$(x, y)$。此方程之所以无解是因为左式可写作$2(x + 2y)$,即$2$的倍数,而右式$5$却不是$2$的倍数,故此方程无整数解。

\paragraph{一个会漏掉解的例子}
	从以下的二元一次不定方程
	\[ 2x + 4y = 6 \]
	可看出$(x_0, y_0) = (1, 1)$是其中一个解。接下来我们根据之前讨论了的方法,试图找出所有其它的解,即我们利用
	\[ \begin{dcases}
			x = x_0 + 4t \\
			y = y_0 - 2t
		\end{dcases}
	\]
	生成以下一系列的解:
	\begin{center}
	\begin{tabular}{c|ccccccccc}
		$t$ & $\cdots$ & $-3$ & $-2$ & $-1$ & $0$ & $1$ & $2$ & $3$ & $\cdots$ \\
		\hline
		$x$ & $\cdots$ & $-11$ & $-7$ & $-3$ & $1$ & $5$ & $9$ & $13$ & $\cdots$ \\
		$y$ & $\cdots$ & $7$ & $5$ & $3$ & $1$ & $-1$ & $-3$ & $-5$ & $\cdots$ \\
	\end{tabular}
	\end{center}
	但这一次的表格并不完整。比如我们可以找到一些漏掉的解,像是最明显的$(3, 0)$。之所以会出现“漏掉解”的情况是因为原方程左右两式有共同因数。解决方法很简单,就是要记得先尽可能把两式的共同因数抵消后,再使用上述规律求得剩余的所有解,即
	\[ \begin{aligned}
			2x + 4y &= 6 \\
			x + 2y &= 3 \ \mbox{。}
		\end{aligned}
	\]
	因此对于任意整数$t$,
	\[ \begin{dcases}
			x = x_0 + 4t \\
			y = y_0 - 2t
		\end{dcases}
	\]
	都会是原方程$2x + 4y = 6$的解。整理成表,则有
	\begin{center}
	\begin{tabular}{c|ccccccccc}
		$t$ & $\cdots$ & $-3$ & $-2$ & $-1$ & $0$ & $1$ & $2$ & $3$ & $\cdots$ \\
		\hline
		$x$ & $\cdots$ & $-5$ & $-3$ & $-1$ & $1$ & $3$ & $5$ & $7$ & $\cdots$ \\
		$y$ & $\cdots$ & $4$ & $3$ & $2$ & $1$ & $0$ & $-1$ & $-2$ & $\cdots$ \\
	\end{tabular}
	\end{center}

\paragraph{二元一次不定方程的公式解}
	现在我们将从前几页讨论的例子中总结归纳出个“公式”。

	考虑以下不定方程
	\[ ax + by = c \ \mbox{,} \]
	其中$a, b, c\in\mathbb{Z}$,而未知数$x$和$y$必须是整数。当$c$为$a$和$b$的最大公因数(greatest common divisor),$\mathrm{gcd}(a, b)$,的倍数时,则不定方程有解。反之,即$\mathrm{gcd}(a, b)$无法整数$c$,则不定方程无解。

	举几个例子。$21x+91y=64$无整数解,因为$\mathrm{gcd}(21, 91) = 7$,然而$7$无法整数$64$。$1001x - 132y= -198$有解,因为$\mathrm{gcd}(1001, -132) = 11$,而$-198$是$11$的倍数。如果$a$和$b$互质,即$\mathrm{gcd}(a, b) = 1$,比如$2x + 3y = c$,那么任意$c\in\mathbb{Z}$,此方程都会有解,因为任何整数$c$都是$\mathrm{gcd}(a, b) = 1$的倍数。

	假设$ax + by = c$有解,并设之为$(x_0, y_0)$,则其它解能通过
	\[ \begin{dcases}
			x = x_0 + \frac{b}{\mathrm{gcd}(a, b)}\cdot t \\
			y = y_0 - \frac{a}{\mathrm{gcd}(a, b)}\cdot t \\
		\end{dcases}
	\]
	产生而得,其中$t$为任意整数,并且每一个$t$值都对应了一组不同的解。当然,如果事先把$ax + by = c$左右两式最大共同因数消掉,那这个“公式”就变成简洁的
	\[ \begin{dcases}
			x = x_0 + bt \\
			y = y_0 - at \\
		\end{dcases} \ \mbox{。}
	\]
	举个例子,$21x + 91y = -56$,应先化简为$3x + 13y = -8$,再使用以上的“公式”。

	最后我们给这方法取个非正式名字,方便日后的讨论。由于上述结论的推导源自
	\[ ax + by + (ab - ab)t = c \ \mbox{,} \]
	作者一般管它叫“加$ab$减$ab$公式”。

\paragraph{辗转相除法}
	本文最后一个内容就是探讨如何有系统地找出第一组解$(x_0, y_0)$。细心的同学可能已经发现作者至今每每提及$(x_0, y_0)$都是用“猜”的。但这一招显然无法招架数字组合的万千变化。我们可以拿上一段最后一个例子,$3x + 13y = -8$,问读者能否快速看出解呢?如果能,那$376x - 157y = 101$呢?

	在检查不定方程是有解之后,我们便演示如何通过辗转相除法为不定方程先求得一个解,之后再用之前的“公式”把其它解一并列出。以
	\[ 3x + 13y = -8 \]
	为例,在$\abs{a}=3$和$\abs{b}=13$当中选一个大的,然后以大的除以小的,并保留余数:
	\[ 13 \divisionsymbol 3 = 4~\mbox{余}~1 \ \mbox{。} \]
	另一个更适合的写法为,
	\[ 13 = 4(3) + 1 \ \mbox{。} \]
	这就是辗转相除法,它的运算止步于当余数为一,因为一旦有了如上的等式,我们便可以得到
	\[ \begin{aligned}
			13 - 4(3) &= 1 \\
			3(-4) + 13 &= 1 \\
			3[(-4)\times(-8)] + 13[1\times(-8)] &= -8 \\
			3(32) + 13(-8) &= 8 \ \mbox{,}
		\end{aligned}
	\]
	即$(x_0, y_0) = (32, -8)$是不定方程的其中一个解。

	第一个例子让人有些意犹未尽,毕竟我们只看到了“相除”,少了“辗转”。所以我们再来看下第二个例子,
	\[ 376x - 157y = 101 \ \mbox{。} \]
	首先我们检查发现$376$和$157$互质,因此不定方程有无限个解,且我们不需要左右两式抵消共同因数。接下来我们使用辗转相除法:
	\[ \begin{aligned}
			376 &= 2(157) + 62 \\
			157 &= 2(62) + 33 \\
			62 &= 1(33) + 29 \\
			33 &= 1(29) + 4 \\
			29 &= 7(4) + 1 \ \mbox{。}
		\end{aligned}
	\]
	当余数为一,我们便停止“辗转”。现在,我们从最后一行开始,把$1$表达为$376X-157Y$的形式:
	\[ \begin{aligned}
			1
			&= 29 - 7(4) \\
			&= 29 - 7(33 - 1(29)) \\
			&= -7(33) + 8(29) \\
			&= -7(33) + 8(62 - 1(33)) \\
			&= -15(33) + 8(62) \\
			&= -15(157 - 2(62)) + 8(62) \\
			&= -15(157) + 38(62) \\
			&= -15(157) + 38(376 - 2(157)) \\
			&= -91(157) + 38(376) \\
			\therefore 1 &= 376(38) - 157(91) \ \mbox{。}
		\end{aligned}
	\]
	作者把这过程非正式地称为“辗转代入法”。其过程就是不断使用辗转相除法过程中的每一行,并逐步替换掉括号里数字,直到有了$1=376X-157Y$,我们变能算得原方程$376x - 157y = 101$的解:
	\[ \begin{aligned}
			376(38\times101) - 157(91\times101) &= 101 \\
			376(3838) - 157(9191) &= 101 \ \mbox{。}
		\end{aligned}
	\]
	因此,我们得到$(x_0, y_0) = (3838, 9191)$。剩下的解,则可通过“加$ab$减$ab$公式”一一求得。

\paragraph{上回的题目}
\emph{
	小明身上没钱,便拿了一张有四位数金额的支票到银行兑现。糊涂的出纳员将金额的四位数$abcd$看成$cdab$,而给了小明这金额的钱。小明没数就将钱拿走,到超市拿出其中的$\mathrm{RM}50$买了$\mathrm{RM}50$的东西后,才发现剩下的钱刚好等于支票上数额的$3$倍。问小明支票上原来的金额的最后三位数是什么?
	}

	作者将用横杠来区分十进制表达式和数的乘积,即$abcd$是$a\cdot b\cdot c\cdot d$,而$\overline{abcd}$是个千位数为$a$的数字。根据小明的经历,我们便有
	\[ \overline{cdab} - 50 = 3\cdot\overline{abcd} \ \mbox{。} \]
	接着我们把十进制展开,
	\[ \begin{aligned}
			(10^3c + 10^2d + 10a + b) - 50 &= 3(10^3a + 10^2b + 10c + d) \\
			2990a + 299b - 970c -97d &= -50 \ \mbox{。}
		\end{aligned}
	\]
	这是一个四元一次不定方程。但我们可把它化成二元一次:
	\[ \begin{aligned}
			299(10a + b) - 97(10c + d) &= -50 \\
			299x - 97y &= -50 \ \mbox{。}
		\end{aligned}
	\]
	这里我们设$x:= 10a+b$和$y:= 10c+d$。由于$a, b, c, d$皆为$0, \ldots, 9$的个位数,因此$x$和$y$的取值范围是$10, \ldots, 99$。接下来,我们把刚才所学给用起来。

	首先,$\mathrm{gcd}(299, 97) = 1$,因此$299x - 97y = -50$有整数解。接下来,我们利用辗转相除法:
	\[ \begin{aligned}
			299 &= 3(97) + 8 \\
			97 &= 12(8) + 1 \ \mbox{。}
		\end{aligned}
	\]
	再用“辗转代入法”,得
	\[ \begin{aligned}
			1
			&= 97 - 12(8) \\
			&= 97 - 12(299 - 3(97)) \\
			&= 37(97) - 12(299) \ \mbox{。}
		\end{aligned}
	\]
	因此,不定方程$299x - 97y = -50$的解可由
	\[ \begin{aligned}
			299(-12) - 97(-37) &= 1 \\
			299[(-12)\times(-50)] - 97[(-37)\times(-50)] &= -50 \\
			299(600) - 97(1850) &= -50
		\end{aligned}
	\]
	得$(x_0, y_0) = (600, 1850)$。

	然而并不是所有解都能满足题目,因为$x$和$y$的取值范围必须是$10, \ldots, 99$。为此我们从$(x_0, y_0) = (600, 1850)$开始生成更小的解。“加$ab$减$ab$公式”告诉我们其它解为
	\[ \begin{dcases}
			x = x_0 + (-97)t \\
			y = y_0 - 299t
		\end{dcases}
		\Longrightarrow
		\begin{dcases}
			x = 600 - 97t \\
			y = 1850 - 299t
		\end{dcases} \ \mbox{。}
	\]
	而为了让$x$和$y$都介于$10, \ldots, 99$,不难发现$t=6$是唯一能满足此条件的整数参数。因此算得
	\[ \begin{dcases}
			x = 18 \\
			y = 56
		\end{dcases} \ \mbox{。}
	\]
	也就是说$\overline{abcd} = 1856$。我们可以做最后的验算:
	\[ 3\times1856 = 5618 - 50 = 5568 \ \mbox{。} \]
	故,小明支票上原来金额的最后三位数是$856$。
	\Qed

\end{document}
