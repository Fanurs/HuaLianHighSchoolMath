\documentclass[a4,12pt]{article}
	\usepackage[UTF8]{ctex}
	\usepackage{datetime2}% to use \DTMnow
	\usepackage{indentfirst}
		\setlength{\parskip}{5pt}% spacing between paragraphs
		\setlength{\parindent}{20pt}% indentation
	\usepackage{geometry}
		\geometry{tmargin=8mm,bmargin=15mm,lmargin=8mm,rmargin=8mm}
	\usepackage{enumitem}
	\usepackage{amsmath}
	\usepackage{amssymb}% use mathcal, etc.
	\usepackage{mathtools}
	\usepackage{physics}
	\usepackage{tabu}
	\usepackage{tikz}
		\usetikzlibrary{arrows.meta,calc,decorations.markings,math,arrows.meta,patterns,angles,quotes,decorations.pathreplacing}

% self-defined commands always begin with UPPERCASE LETTER
	\newenvironment{LatinModern}{\fontfamily{lmr}\selectfont}{\par}% Latin Modern
	\DeclareTextFontCommand{\textmyfont}{\LatinModern}
	\newcommand{\Curs}[1]{\emph{\LatinModern{#1}}}
	\newcounter{Problem}
	\newcommand{\Problem}[1]{
		\vspace*{10pt}
		\stepcounter{Problem}
		\label{Problem \arabic{Problem}}
		\noindent\arabic{Problem}.\emph{~#1}
	}
	\newcommand{\Qed}{\hfill\ensuremath{\square}}

% useful symbols
	\def\Tick{\ding{51}}
	\newcommand{\rectangle}{{
		\ooalign{$\sqsubset\mkern3mu$\cr$\mkern3mu\sqsupset$\cr}
	}}

\begin{document}

% Title
	\title{
		\vspace*{-50pt}
		\Large{\textbf{太平华联中学高三数学题解}\vspace*{-10pt}}\\
	}
	\author{作者:郑其恩 Fanurs\vspace*{-30pt}}
	\date{最后编译时间:\DTMnow (美东)}
	\maketitle
	\vspace*{-25pt}

\setcounter{Problem}{0}
\textbf{提问同学:某某某}\\
\Problem{
	求导$f(x)=x^x$。
	}

	此题是隐函数微分的经典应用。

	首先设$y:=x^x$,然后我们对左右两式取自然对数,得
	\[\ln{y} = x\ln{x}\]

	记得我们的目标是求得$f'(x)$,即$\frac{dy}{dx}$。因此针对上式,左右两式求导,
	\[\frac{d}{dx}\ln{y} = \frac{d}{dx}x\ln{x}\]

	右式,利用导数的乘积法则可化为$(x\ln{x})' = x\cdot\frac{1}{x} + \ln{x}$。由于$f(x) = x^x$的定义域最多只能是$x\in\mathbb{R}^+$,即$x$不可为零或负数,因此右式最终可写成$1 + \ln{x}$。

	左式,我们利用链导法则,
	\[\frac{d}{dx}\ln{y} = \frac{dy}{dx}\left(\frac{d}{dy}\ln{y}\right) = y'\cdot\frac{1}{y}\]

	由此,比较左右两式变可得
	\[y' = y\left(1 + \ln{x}\right)\]

	最后记得把$y$全部换掉,答案应写作
	\[f'(x) = x^x \left(1 + \ln{x}\right)\]
	\Qed
	\vspace*{30pt}

	\noindent 点评:
	\begin{enumerate}[label=(\alph*)]
		\item 作为练习,同学可尝试求导$f(x) = x^{x^x}$。
	\end{enumerate}


\end{document}
